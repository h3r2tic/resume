%% start of file `letter.tex'.
%% Copyright 2006-2010 Xavier Danaux (xdanaux@gmail.com).
%
% This work may be distributed and/or modified under the
% conditions of the LaTeX Project Public License version 1.3c,
% available at http://www.latex-project.org/lppl/.


\documentclass[11pt]{article}

\setlength\paperheight{297mm}
\setlength\paperwidth{210mm}

%\usepackage[utf8x]{inputenc}
\usepackage[T1]{fontenc}
\usepackage{lmodern}
\usepackage{marvosym}
\usepackage{ifpdf}
\ifpdf
  \usepackage[pdftex]{graphicx}
\else
  \usepackage[dvips]{graphicx}\fi

\pagestyle{empty}

\usepackage[scale={0.8, 0.82}, noheadfoot]{geometry}
\setlength{\parindent}{0pt}
\addtolength{\parskip}{6pt}

\linespread{1.0925}

\usepackage{xltxtra}
\setmainfont[Mapping=tex-text,Ligatures={Common}]{Palatino Linotype}

\def\firstname{Tomasz}
\def\familyname{Stachowiak}
\def\FileAuthor{\firstname \familyname}
\def\FileTitle{\firstname \familyname's cover letter}
\def\FileSubject{Cover letter}
\def\FileKeyWords{\firstname \familyname, Cover letter}

\usepackage{url}
\renewcommand{\ttdefault}{pcr}
\urlstyle{tt}
\ifpdf
  \usepackage[pdftex,pdfborder=0,breaklinks,baseurl=http://,pdfpagemode=None,pdfstartview=XYZ,pdfstartpage=1]{hyperref}
  \hypersetup{
    pdfauthor   = \FileAuthor,%
    pdftitle    = \FileTitle,%
    pdfsubject  = \FileSubject,%
    pdfkeywords = \FileKeyWords,%
    pdfcreator  = \LaTeX,%
    pdfproducer = \LaTeX}
\else
  \usepackage[dvips]{hyperref}
\fi

%\renewcommand{\familydefault}{\sfdefault}% for use with a résumé using sans serif fonts;
%\renewcommand{\familydefault}{\rmdefault}% for use with a résumé using sans serif fonts;

\begin{document}
\hfill%
\begin{minipage}[t]{.6\textwidth}
\raggedleft%
{\bfseries Tomasz Stachowiak}\\[.35ex]
\small\itshape%
23 Barbara Road\\
Leicester LE3 2EB, UK\\[.35ex]
\Telefon~+44 (0) 785 628 92 84\\
\Letter~\href{mailto:h3@h3.gd}{h3@h3.gd}
\end{minipage}\\[1em]
%
\begin{minipage}[t]{.4\textwidth}
\raggedright%
{\bfseries Ubisoft Massive}\\[.35ex]
\small\itshape%
Drottninggatan 34\\
203 14 Malmö, Sweden
\end{minipage}
\hfill % US style
%\\[1em] % UK style
\begin{minipage}[t]{.4\textwidth}
\raggedleft % US style
\today
%April 6, 2006 % US informal style
%05/04/2006 % UK formal style
\end{minipage}\\[2em]
%\vfill
%\raggedright
%\flushleft
Dear Sir or Madam,\\[2.5em]
%\vfill
I am writing to apply for a position as a Tech and Tools Programmer, referring to the job announcement posted in the \emph{Available Positions} section of the website.

Being a fan of the original \emph{Ground Control} game, I had no second thoughts about applying at \emph{Massive}. I recognize your company as a creative and successful game developer, and would love to join your ranks, in order to utilize my qualifications by creating even better games from within your team.

For the past 10 years I have worked very hard and exerted a humongous effort in order to achieve my life-long dream of becoming a professional game developer. I have never allowed obstacles to let me down, instead leveraging them as stepping stones. The result have been games and game engines --- ranging from core subsystems, rendering, networking, to physics and gameplay. I also made custom tools, such as mesh exporters and code profilers. My recent work has culminated in an MSc thesis on a flexible real-time rendering system bearing many similarities to Renderman.

I believe this array of experience will allow me to swiftly integrate myself with your team, where I can learn even more, and through teamwork, approach assignments with a clear vision of a solution.

Despite having no professional experience working in the game industry, I have been able to learn the invaluable team co-operation, scheduling and mentoring skills. Within the course of my studies I served the role of a lead programmer of a team of 5. We managed to create \emph{Deadlock}, a networked first person shooter game and its underlying engine. The resulting, virtually bug-free product has so far been the largest in the history of the ``Team Programming'' course.

I've always had a knack for implementing creative solutions and thinking outside of the box. Upon getting my first camera phone, I created a pipeline for snapping HDR photos and tone mapping them. When I learned about templating in the D language, I created a compile-time ray-tracer.

I love knowledge sharing, which I learned by successfully organizing university workshops and presentations, and recently during two talks at the 2008 Tango D conference. Selfless attitude and a helpful spirit --- towards close friends and complete strangers alike --- is what I try to exhibit and appreciate among people in general; this is the very essence of teamwork. As a sign of gratitude, by helping others, I have received cookies via intercontinental air mail on two occasions already!

I am looking forward to the opportunity of meeting you and learning more about the position, your objectives and how I can help you create the best games in the industry.


%\vfill  
%Yours sincerely,\\[2em] % if the opening is "Dear Mr(s) Doe,"
Yours faithfully,\\[2em] % if the opening is "Dear Sir or Madam,"
%\vfill
%
%\includegraphics[scale=0.75]{signature_blue}\\
{\bfseries Tomasz Stachowiak}
%\\
%
%\vfill%
%{\slshape Enclosure}
%{\slshape Attachment: curriculum vit\ae{}}
\end{document}
