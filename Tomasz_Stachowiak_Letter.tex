%% start of file `letter.tex'.
%% Copyright 2006-2010 Xavier Danaux (xdanaux@gmail.com).
%
% This work may be distributed and/or modified under the
% conditions of the LaTeX Project Public License version 1.3c,
% available at http://www.latex-project.org/lppl/.


\documentclass[11pt]{article}

\setlength\paperheight{297mm}
\setlength\paperwidth{210mm}

%\usepackage[utf8x]{inputenc}
\usepackage[T1]{fontenc}
\usepackage{lmodern}
\usepackage{marvosym}
\usepackage{ifpdf}
\ifpdf
  \usepackage[pdftex]{graphicx}
\else
  \usepackage[dvips]{graphicx}\fi

\pagestyle{empty}

\usepackage[scale=0.8]{geometry}
\setlength{\parindent}{0pt}
\addtolength{\parskip}{6pt}

\linespread{1.0925}

\usepackage{xltxtra}
\setmainfont[Ligatures={Common}]{Palatino Linotype}

\def\firstname{Tomasz}
\def\familyname{Stachowiak}
\def\FileAuthor{\firstname \familyname}
\def\FileTitle{\firstname \familyname's cover letter}
\def\FileSubject{Cover letter}
\def\FileKeyWords{\firstname \familyname, Cover letter}

\usepackage{url}
\renewcommand{\ttdefault}{pcr}
\urlstyle{tt}
\ifpdf
  \usepackage[pdftex,pdfborder=0,breaklinks,baseurl=http://,pdfpagemode=None,pdfstartview=XYZ,pdfstartpage=1]{hyperref}
  \hypersetup{
    pdfauthor   = \FileAuthor,%
    pdftitle    = \FileTitle,%
    pdfsubject  = \FileSubject,%
    pdfkeywords = \FileKeyWords,%
    pdfcreator  = \LaTeX,%
    pdfproducer = \LaTeX}
\else
  \usepackage[dvips]{hyperref}
\fi

%\renewcommand{\familydefault}{\sfdefault}% for use with a résumé using sans serif fonts;
%\renewcommand{\familydefault}{\rmdefault}% for use with a résumé using sans serif fonts;

\begin{document}
\hfill%
\begin{minipage}[t]{.6\textwidth}
\raggedleft%
{\bfseries Tomasz Stachowiak}\\[.35ex]
\small\itshape%
23 Barbara Road\\
Leicester LE3 2EB, UK\\[.35ex]
\Telefon~+44 (0) 785 628 92 84\\
\Letter~\href{mailto:h3@h3.gd}{h3@h3.gd}
\end{minipage}\\[1em]
%
\begin{minipage}[t]{.4\textwidth}
\raggedright%
{\bfseries Ubisoft Massive}\\[.35ex]
\small\itshape%
Drottninggatan 34\\
203 14 Malmö, Sweden
\end{minipage}
\hfill % US style
%\\[1em] % UK style
\begin{minipage}[t]{.4\textwidth}
\raggedleft % US style
\today
%April 6, 2006 % US informal style
%05/04/2006 % UK formal style
\end{minipage}\\[2em]
%\raggedright
%\flushleft
Dear Sir or Madam,\\[1.5em]
%
I am writing to apply for a position as a Tech and Tools Programmer, referring to the job announcement posted in the \emph{Available Positions} section of the website. Being a passionate and experienced programmer, I am confident of the benefit my skills can bring to the team.

Developing games and game technology has always been my dream, one towards which I have exerted a humongous effort. During the past 10 years I have explored most aspects of game and engine programming, ranging from core subsystems, geometric algorithms, rendering, sound, to networking, physics, and gameplay. Besides core game technology, I have worked on custom tools, such as mesh exporters, processors and code profilers.

I have always been exceptionally self-motivated and apt at learning on my own. Besides a gentle kick-start at the age of 10, I have taught myself most of what I know about software development from industry publications, and by tackling ever more challenging projects. Time after time, I have triumphed at conquering seemingly impossible challenges by successfully breaking them down, conducting research and iterating towards the final solution.

Within the course of my studies I was a lead programmer of a team of 5 programmers, with whom we created \emph{Deadlock}, a networked first person shooter game and its underlying engine. During this experience I have been able to learn not only invaluable team co-operation skills, but had the opportunity to be a mentor for the other team members. This project also stressed my ability to mediate between programmers and university supervisors, whilst working on tight deadlines.

I've always had a knack for implementing creative solutions and thinking outside of the box. Upon getting my first camera phone, I created a scripted pipeline for snapping HDR photos and tone mapping. When I learned about templating in the D language, I created a compile-time ray-tracer.

% As an avid fan of the Battlefield series of games, I believe I would have a great time working on its underlying technology, with which I was able to acquaint myself via numerous conference presentations. Frostbite is exactly the engine I would love to help develop, especially considering that my own hobby programming has been concentrated around similar technical challenges. My latest areas of research have been networked physics simulation, as well as shader management and authoring, both of which are crucial factors making Frostbite the strong contender that it is.

% Due to the multitude of subjects which I delved into, I believe I will not have trouble jumping into most of DICE's technology. Still, I have deep understanding of the fields which interest me the most, acquired through implementing software considered arcane by many programmers, such as software rasterization.

I love sharing my knowledge, which I practiced successfully by organizing university workshops and presentations, and recently during two talks at the 2008 Tango D conference. I am also known for my selfless attitude and a helpful spirit, towards close friends and complete strangers alike. As a sign of gratitude, I have received cookies via air mail on two occasions already!

I am looking forward to the opportunity of meeting you and learning more about the position, your objectives and how I can contribute to the success of your company.
  
%Yours sincerely,\\[2em] % if the opening is "Dear Mr(s) Doe,"
Yours faithfully,\\[2em] % if the opening is "Dear Sir or Madam,"
%
%\includegraphics[scale=0.75]{signature_blue}\\
{\bfseries Tomasz Stachowiak}
%\\
%
%\vfill%
%{\slshape Enclosure}
%{\slshape Attachment: curriculum vit\ae{}}
\end{document}
