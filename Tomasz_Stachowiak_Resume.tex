%% start of file `template.tex'.
%% Copyright 2006-2010 Xavier Danaux (xdanaux@gmail.com).
%
% This work may be distributed and/or modified under the
% conditions of the LaTeX Project Public License version 1.3c,
% available at http://www.latex-project.org/lppl/.


\documentclass[11pt,a4paper]{moderncv}

% moderncv themes
%\moderncvtheme[blue]{casual}                 % optional argument are 'blue' (default), 'orange', 'red', 'green', 'grey' and 'roman' (for roman fonts, instead of sans serif fonts)


\definecolor{quotecolor}{rgb}{0.35,0.35,0.35}
\moderncvstyle{casual}                % idem
\moderncvcolor{blue}
%\renewcommand{\familydefault}{\sfdefault}
%\setmainfont[Mapping=tex-text,Ligatures={Common}]{Palatino Linotype}
\setmainfont[Mapping=tex-text,Ligatures={Common}]{Segoe UI}

% character encoding
%\usepackage[utf8]{inputenc}                   % replace by the encoding you are using

% adjust the page margins
\usepackage[scale=0.8]{geometry}
%\setlength{\hintscolumnwidth}{3cm}						% if you want to change the width of the column with the dates
%\AtBeginDocument{\setlength{\maketitlenamewidth}{6cm}}  % only for the classic theme, if you want to change the width of your name placeholder (to leave more space for your address details
%\AtBeginDocument{\recomputelengths}                     % required when changes are made to page layout lengths

\usepackage[none]{hyphenat}
%\raggedright

% personal data
\firstname{Tomasz}
\familyname{Stachowiak}
%\title{Resumé title (optional)}               % optional, remove the line if not wanted
\title{Engine Programmer}
\address{3 Trenear Close}{Horsham RH13 5UP, United Kingdom}    % optional, remove the line if not wanted
\mobile{+44 (0) 74 111 47 291}                   % optional, remove the line if not wanted
\email{h3@h3.gd}                      % optional, remove the line if not wanted
\homepage{http://h3.gd}                % optional, remove the line if not wanted
% \photo[64pt]{lastfm_mod.png}                         % '64pt' is the height the picture must be resized to and 'picture' is the name of the picture file; optional, remove the line if not wanted
\quote{Passion for game development;\\Fascination with new and unsolved challenges;\\Deep desire to learn and share knowledge.}                 % optional, remove the line if not wanted

% to show numerical labels in the bibliography; only useful if you make citations in your resume
\makeatletter
\renewcommand*{\bibliographyitemlabel}{\@biblabel{\arabic{enumiv}}}
\makeatother

%\usepackage{enumitem}
\makeatletter
\newcommand\novspace{\@minipagetrue}
\makeatother

%\renewcommand*{\cvline}[3][.25em]{%
%  \begin{tabular}{@{}p{\hintscolumnwidth}@{\hspace{\separatorcolumnwidth}}p{\maincolumnwidth}@{}}%
% 	  \raggedleft\hintfont{#2} &{\novspace #3}%
%  \end{tabular}\\[#1]}

\newcommand{\projecturl}[1]{\textnormal{\hspace{0.5em}\color{quotecolor}\small(\href{#1}{#1})}}

% bibliography with mutiple entries
%\usepackage{multibib}
%\newcites{book,misc}{{Books},{Others}}

%\nopagenumbers{}                             % uncomment to suppress automatic page numbering for CVs longer than one page
%----------------------------------------------------------------------------------
%            content
%----------------------------------------------------------------------------------
\begin{document}
\maketitle

\section{Areas of expertise}
\cvline{$\Longrightarrow$}{\small Design and implementation of robust engine components;}
\cvline{$\Longrightarrow$}{\small Physically-based rendering and global illumination;}
\cvline{$\Longrightarrow$}{\small Development of cache- and cycle-efficient algorithms;}
\cvline{$\Longrightarrow$}{\small Engineering of multi-threaded and distributed systems;}
\cvline{$\Longrightarrow$}{\small Crafting artist-friendly tools and pipelines.}

\section{Work experience}

%\vspace{0.07in}

\cventry{Apr~2011 -- Present}{Engine Programmer}{Creative Assembly}{Alien: Isolation}{}{Architected and implemented algorithms for PS3, PS4, XBox 360, XBox One, and PC:%
\begin{itemize}%
\item Real-time radiosity pipeline, runtime, and tools;
\item Physically-based rendering;
\item Skin and hair shading, wrinkle and morph animation;
\item HDR, tone-mapping, post-processing;
\item Research, optimization, code maintenance, artist support, miscellaneous features.
\end{itemize}}

\section{Education}
\cventry{2004--2010}{MSc, Computer Science}{Nicolaus Copernicus University}{Toruń, Poland}{}{}  % arguments 3 to 6 can be left empty
\cvline{Relevant courses}{\small Linear Algebra, Differential Equations, Network Programming, Numerical Methods, Parallel and Distributed Programming, Stochastic Simulations, Team Programming.}

\section{Master's thesis \projecturl{http://h3.gd/pub/msc.html}}
\cvline{Title}{\emph{A flexible system for real-time rendering of three-dimensional scenes}}
\cvline{Description}{\small Presented a rendering architecture which facilitates shader reuse between algorithms, and prevents errors via a semantic type system; RenderMan-like shader stages.}

\section{Personal projects}

\cventry{2013 -- Present}{Shuriken \projecturl{http://h3.gd/TODO}}{}{}{}{Graphics research playground:
\begin{itemize}
	\item Physically-based rendering, real-time reflections;
	\item Using OpenGL 4 and the Rust programming language.
\end{itemize}}

\cventry{2008 -- 2010}{Boxen \projecturl{http://h3.gd/TODO}}{}{}{}{Multi-player network programming playground:
\begin{itemize}
	\item Physics synchronization for characters and complex vehicles.
\end{itemize}}

\cventry{Oct~2007 -- Apr~2008}{Deadlock \projecturl{http://h3.gd/code/deadlock/}}{}{}{}{Multi-player FPS game and engine done for a university course:%
\begin{itemize}%
\item Management and mentoring of a 5-person team;
\item Programming:
  \begin{itemize}%
  \item Graphics rendering;
  \item Network match-making and synchronization;
  \item Core engine components, tools.
  \end{itemize}
\end{itemize}}

\cventry{2007 -- 2008}{Hybrid \projecturl{http://h3.gd/code/hybrid/}}{}{}{}{Immediate-Mode GUI toolkit for OpenGL:
\begin{itemize}
	\item Combining the benefits of immediate- and retained-mode interfaces;
	\item Efficient sub-pixel rendering of text and widgets.
\end{itemize}}

\cventry{Mar 2007}{DShade \projecturl{http://h3.gd/TODO}}{}{}{}{Software 3D renderer:
\begin{itemize}
	\item Vertex and pixel shaders on the CPU;
	\item Clipping, perspective projection, texturing, Z-buffering, blending.
\end{itemize}}

\section{Workshops and conferences}
\cventry{Jul~2012}{Develop Conference}{}{Brighton, UK \projecturl{http://h3.gd/TODO}}{}{Detailed multi-BRDF rendering via stencil masking for the PS3 and XBox 360.}%
\cventry{Sep~2008}{Tango Conference}{Nicolaus Copernicus University}{Toruń, Poland}{}{Held talks about run-time linking and game programming for an international audience.}%
\cventry{2007--2008}{D Workshops}{Nicolaus Copernicus University}{Toruń, Poland}{}{Organized weekly hands-on courses on the D programing language and OpenGL.}%

\section{Languages}
\cvlanguage{Polish}{Native}{}
\cvlanguage{English}{Fluent}{}
\cvlanguage{German}{Basic}{}

\section{Skills and interests}
\cvline{Programming}{Excellent knowledge of C/C++, HLSL, GLSL, D 1.0, Python.\newline{}
Working knowledge of Mathematica, MaxScript, x86 Assembly, Rust, C\#, Java.}
\cvline{API}{OpenGL 4, libgcm, DX9, DX11, FreeType, FreeImage, GLFW, FMOD, ENet, ...}
\cvline{Graphics}{Climbing the Uncanny Valley, physically-based rendering, path-tracing, real-time global illumination, shader systems.}
\cvline{Networking}{Networked physics and gameplay synchronization with authority schemes.}
\cvline{Misc}{Memory management, profiling, linking, domain-specific languages.}

\section{Activities}
\cvline{Sports}{Inline and ice skating, swimming, running, martial arts.}
\cvline{Games}{Battlefield, Mass Effect, Dragon Age, League of Legends.}

% Publications from a BibTeX file without multibib\renewcommand*{\bibliographyitemlabel}{\@biblabel{\arabic{enumiv}}}% for BibTeX numerical labels
\nocite{*}
\bibliographystyle{plain}
\bibliography{publications}       % 'publications' is the name of a BibTeX file

% Publications from a BibTeX file using the multibib package
%\section{Publications}
%\nocitebook{book1,book2}
%\bibliographystylebook{plain}
%\bibliographybook{publications}   % 'publications' is the name of a BibTeX file
%\nocitemisc{misc1,misc2,misc3}
%\bibliographystylemisc{plain}
%\bibliographymisc{publications}   % 'publications' is the name of a BibTeX file

%\color{color2}\addressfont{\today}
\end{document}


%% end of file `template_en.tex'.
